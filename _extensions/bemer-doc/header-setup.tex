\usepackage{lastpage}
\usepackage{datatool}
\usepackage{longtable,tabularx,ltxtable}
\usepackage{array,multirow,multicol,booktabs,hhline}
\usepackage{colortbl}
\usepackage{pifont}
\usepackage{graphicx,lastpage}
\usepackage[export]{adjustbox}
\usepackage{xstring}
\usepackage[useregional]{datetime2}
\usepackage{ragged2e}
\usepackage{tabto}
\usepackage{enumitem}
\usepackage[normalem]{ulem}
\usepackage{amsmath}
\usepackage{eurosym}
\usepackage{iflang}
\usepackage{ifthen}
\usepackage{fp}
\usepackage{pgf,pgfplotstable}
%------------------------------------------------
\IfLanguageName{french}{\frenchbsetup{StandardLayout=true}}{}
\def\labelitemi{---}
\def\labelitemii{--}
\setenumerate{itemsep=0cm}
\setlist{itemsep=0cm}
\setlength{\parindent}{0cm}
\definecolor{bemer}{RGB}{255,148,26}
\definecolor{bemertrue}{RGB}{244,121,32}
\definecolor{bgray}{RGB}{92,98,98}
\definecolor{lgray}{RGB}{245,244,243}
\definecolor{bblue}{RGB}{59,98,148}
\newcolumntype{C}{>{\centering\arraybackslash}X}
\newcolumntype{R}[1]{>{\RaggedLeft\arraybackslash}p{#1}}
\newcolumntype{L}[1]{>{\RaggedRight\arraybackslash}p{#1}}
%------------------------------------------------
\usepackage{mdframed}
\newmdenv[
  topline=false,
  bottomline=false,
  rightline=false,
  backgroundcolor=lgray,
  linecolor=bemertrue,
  linewidth=.13cm,
  skipabove=0pt,
  skipbelow=.3cm,
  leftmargin=0pt,
  rightmargin=0pt,
  innertopmargin=.3cm,
  innerbottommargin=.3cm,
  innerleftmargin=.8cm,
  innerrightmargin=.5cm,
]{combox}
\newmdenv[
  topline=false,
  bottomline=true,
  leftline=false,
  rightline=false,
  backgroundcolor=white,
  linecolor=black,
  linewidth=.01cm,
  skipabove=0pt,
  skipbelow=0pt,
  leftmargin=0pt,
  rightmargin=0pt,
  innertopmargin=.1cm,
  innerbottommargin=.1cm,
  innerleftmargin=.2cm,
  innerrightmargin=.5cm,
]{cfield}
\newmdenv[
  topline=false,
  bottomline=false,
  leftline=false,
  rightline=false,
  backgroundcolor=white,
  linecolor=black,
  linewidth=.01cm,
  skipabove=0pt,
  skipbelow=0pt,
  leftmargin=0pt,
  rightmargin=0pt,
  innertopmargin=.1cm,
  innerbottommargin=.1cm,
  innerleftmargin=.2cm,
  innerrightmargin=.5cm,
]{rfield}

\newcommand{\bField}[1]{\TextField[name=#1,
		   width=\linewidth,
		   bordercolor=,
		   backgroundcolor=,
           format = {
               var f = this.getField('#1');
               f.textFont = 'Times New Roman';
               f.strokeColor = ['T'];
               f.fillColor = ['T'];
               },
           charsize = 10pt]
          {\null}}
          
\newcommand{\bFieldR}[1]{\TextField[name=#1,
		   width=\linewidth,
		   bordercolor=,
		   backgroundcolor=,
		   align=2,
           format = {
               var f = this.getField('#1');
               f.textFont = 'Times New Roman';
               f.strokeColor = ['T'];
               f.fillColor = ['T'];
               },
           charsize = 10pt]
          {\null}}
          
\newcommand{\bArea}[2]{\TextField[name=#1,
		   width=\linewidth,
		   bordercolor=,
		   backgroundcolor=,
		   height=#2,
		   multiline=true,
           format = {
               var f = this.getField('#1');
               f.textFont = 'Times New Roman';
               f.strokeColor = ['T'];
               f.fillColor = ['T'];
               },
           charsize = 10pt]
          {\null}}
          
\newcommand{\bCheck}[1]{\raisebox{.1cm}{%
			\CheckBox[name=#1,
			   width=.2cm,
			   height=.2cm,
			   bordercolor=black,
			   backgroundcolor=]{\null}}}